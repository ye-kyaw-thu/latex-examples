% latex document မှာ မြန်မာစာ၊ ထိုင်း၊ ခမာ နဲ့ CJK စာတွေကို ပေါ်ဖို့ ဘယ်လို လုပ်ရသလဲ ဆိုတဲ့ example latex ပါ
% Ye Kyaw Thu @LST, NECTEC, Thailand
% Last Updated: 28 July 2022

\documentclass[12pt]{article}

\usepackage{fontspec}
\usepackage[utf8]{inputenc}
%\inputencoding{latin1}
\inputencoding{utf8}

% if you want to setup Khmer font as the main font
%\newcommand{\kh}{\setmainfont{Khmer OS}}

% setting the main font of this latex document
\setmainfont{Times New Roman}

% example usage of Courier New and Times New Roman
\newcommand{\courier}{\setmainfont{Courier New}\textbf}
\newcommand{\vsp}{\fontencoding{T3}\fontfamily{cmr}\selectfont\textvisiblespace\setmainfont{Times New Roman}}

%for writing Khmer text
% old approach
%\newfontinstance {\kh}[Script=Khmer]{Khmer OS Siemreap}
%\newfontinstance {\khs}[Script=Khmer,Scale=0.9]{Khmer OS Siemreap}

\newfontfamily {\kh}[Script=Khmer]{Khmer OS Siemreap}
\newfontfamily {\khs}[Script=Khmer,Scale=0.9]{Khmer OS Siemreap}

% for writing Thai text
\newfontfamily {\thaitext}[Script=Thai]{Noto Sans Thai}

%for writing Myanmar text, you can also used with Myanmar3 font
\newfontfamily {\padauktext}[Script=Myanmar]{Padauk}
\newfontfamily {\myanmartext}[Script=Myanmar]{Myanmar3}
%\newfontfamily {\zawgyionetext}[Script=Myanmar]{Zawgyi-One}
%\newfontinstance {\padauktext}[Script=Myanmar]{Padauk}
\newfontfamily {\pdstext}[Script=Myanmar]{Pyidaungsu}

% for Chinese Japanese and Korean
\usepackage{xeCJK}
\setCJKmainfont{UnGungseo.ttf}
\setCJKsansfont{UnGungseo.ttf}
\setCJKmonofont{gulim.ttf}

\usepackage{CJKutf8}

\newcommand{\quotes}[1]{``#1''}

\begin{document}

\section*{Khmer Font Using Example}

For example: //

{\kh - \small{ខ្ញុំ ចង់ឱ្យ \textlangle{} អ្នកស្តាប់ \textrangle{} យល់ ពី បញ្ហា នេះ}}

- I want listener to understand this problem\\

{\kh - \small{ខ្ញុំ ចង់ឱ្យ \textlangle{} អ្នក \textrangle{} \textlangle{} ស្តាប់ \textrangle{} យល់ ពី បញ្ហា នេះ}}

- I want you to listen in order to understand this problem\\

\section*{Thai Font Using Example}

{\thaitext ประเทศไทยคือบ้านหลังที่สามของฉัน}

\section*{Myanmar Font Using Example}

{\padauktext ဗမာစာ၊ ဗမာစကား ကို} \quotes{{\padauktext ပိတောက်ဖောင့်}} {\padauktext ဖြင့် ရိုက်ကြည့်ခြင်း}\\
{\myanmartext ဗမာစာ၊ ဗမာစကား ကို} \quotes{{\padauktext Myanmar3ဖောင့်}}  {\padauktext နဲ့ ရိုက်ကြည့်ခြင်း}\\
{\pdstext ဗမာစာ၊ ဗမာစကား ကို} \quotes{{\padauktext ပြည်ထောင်စုဖောင့်}} {\padauktext ဖြင့် ရိုက်ကြည့်ခြင်း}\\

\section*{Typing Chinese, Japanese and Korean}

我去过北京。\\
京都に行ったことがあります。\\
서울에 다녀왔습니다. \\

\end{document}

